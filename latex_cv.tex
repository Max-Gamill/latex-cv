\documentclass[a4paper,11pt]{article}

% Packages
\usepackage[utf8]{inputenc}
\usepackage[T1]{fontenc}
\usepackage{geometry}
\usepackage{xcolor}
\usepackage{fontawesome5}
\usepackage{tikz}
\usepackage{graphicx}
\usepackage{enumitem}
\usepackage{tabularx}
\usepackage{array}
\usepackage{multicol}
\usepackage{textcomp}
\usepackage{url}
\usepackage{hyperref}

% Page setup
\geometry{margin=0.5in}
\pagestyle{empty}

% Colours
\definecolor{primarycolour}{RGB}{220, 57, 53}%red{36,123,160}cerulean
\definecolor{darkgray}{RGB}{80, 80, 80}
\definecolor{lightgray}{RGB}{150, 150, 150}
\definecolor{backgroundgray}{RGB}{245, 245, 245}

% Custom commands for skill bars
\newcommand{\skillbar}[2]{%
    \begin{tikzpicture}[baseline]
        \fill[lightgray] (0,0) rectangle (5,0.15);
        \fill[black] (0,0) rectangle (#2*5/5,0.15);
    \end{tikzpicture}%
}

% Section formatting
\newcommand{\cvsection}[1]{%
    \vspace{8pt}
    {\color{primarycolour}\large\bfseries #1}
    \vspace{4pt}
    \hrule
    \vspace{8pt}
}

\newcommand{\cvsubsection}[1]{%
    \vspace{6pt}
    {\color{darkgray}\bfseries #1}
    \vspace{4pt}
}

% Custom list environment
\newenvironment{cvlist}{%
    \begin{itemize}[leftmargin=12pt,itemsep=2pt,parsep=0pt,topsep=4pt]
}{%
    \end{itemize}
}

\begin{document}

% Header with photo and contact info
\begin{minipage}[c]{0.7\textwidth}
    {\Huge\bfseries Max Gamill}\\[8pt]
    {\Large\color{darkgray} PhD Candidate}\\[-10pt]
    
    % Contact circles
    {\color{primarycolour}\Huge\textbullet\textbullet\textbullet}
\end{minipage}
\hfill
\begin{minipage}[c]{0.25\textwidth}
    \raggedleft
    % Placeholder for photo - replace with actual photo
    \includegraphics[width=4.0cm,height=4.0cm]{me.png}
\end{minipage}

\vspace{12pt}


% PAGE 1
\begin{minipage}[t]{0.30\textwidth}

% Contact
\cvsection{Contact}
\faPhone\hspace{6pt} +44 7841 907404\\[2pt]
\faEnvelope\hspace{6pt} maxgamill@live.com\\[2pt]
\faGithub\hspace{6pt} MaxGamill

\vspace{12pt}

% Referees
\cvsection{Referees}
\textbf{Alice Pyne}\\
Professor of Biophysics,\\
University of Sheffield\\
\faEnvelope\hspace{6pt} a.l.pyne@sheffield.ac.uk

\vspace{8pt}

\textbf{Mr Neil Shephard}\\
Research Software Engineer\\
University of Sheffield\\
\faEnvelope\hspace{6pt} n.shephard@sheffield.ac.uk

\vspace{8pt}

\textbf{Dr Laura Wiggins}\\
TopoStats Collaborator\\
University of Sheffield\\
\faEnvelope\hspace{6pt} l.wiggins@sheffield.ac.uk

\vspace{12pt}

% Key Skills
\cvsection{Key Skills}
\begin{cvlist}
    \item \textbf{Production software development}. FAIR, modularly designed, Python image analysis pipelines with documentation.
    \item \textbf{Computer vision \& image processing}. Tensorflow to develop, train and adapt segmentation (U-Net, Mask R-CNN), object recognition (YOLO), and generative (CVAE) models for image analysis.
    \item \textbf{CI/CD \& DevOps}. Writing unit / integration / system tests, automating tests and publishing via GitHub actions, and Git version control.
    \item \textbf{Cross-functional collaboration}. Leading diverse teams, liaising with technical and non-technical stakeholders.
\end{cvlist}

\end{minipage}
\hfill
\begin{tikzpicture}[baseline=(current bounding box.north)]
    \draw[lightgray, line width=0.5pt] (0,0) -- (0,-22cm);
\end{tikzpicture}
\hfill
\begin{minipage}[t]{0.62\textwidth}

% Profile
\cvsection{Profile}
My Master's degree in Physics and computational biophysics PhD has focused on computer vision, image analysis, and production-grade software development to quantify DNA interactions. For this, I have developed multiple open-source Python software packages which total $26,000+$ downloads and winning the 2023 Sheffield FAIR software award. This highlights my ability to build modular, reusable tools that bridge cutting-edge AI architectures into practical pipelines. Experienced in collaborative Git workflows, performance optimisation through parallelisation, and CI/CD pipelines via GitHub actions. In my future career I aim to advance my deep learning skills beyond academia and apply computer vision techniques to creative content production and streamlining AI technologies into scalable processing pipelines.

\vspace{12pt}

% Education
\cvsection{Education}

\textbf{Doctorate of Philosophy in Computational Biophysics}

\cvsubsection{University of Sheffield | 2021 - Present | Awarded}
\begin{cvlist}
    \item Developed clustering and computer vision pipelines for molecular segmentation and object recognition of  complex DNA structures.
    \item Refactored Python 2 dependent software into a modular, reusable PyPI package with unit, integration, and system tests, and simple, advanced and function-level documentation.
    \item Collaboration with external contributors and users to provide advice and maintain the software.
    \item Benchmarked software on HPC clusters achieving 5-11x speed-ups.
    \item Created posters and illustration documentation the Adobe suite (Illustrator, InDesign, Photoshop).
    \item Created Blender renderings of biomolecules using their PDB files (atomic coordinates). 
    \item Authored a first-author paper to Nature Communications.
    \item Presented seminars, posters, and workshops at international conferences.
\end{cvlist}

\vspace{8pt}

\textbf{Master of Physics}

\cvsubsection{University of Leeds | 2016 - 2021 | First Class}
\begin{cvlist}
    \item Found a 3D-point predicting deep learning model was more robust to object density and noise vs a mathematical model. This was used to identify the density of specific receptors across a SMLM killer T Cell image.
\end{cvlist}

\end{minipage}

\newpage

% PAGE 2 - Two-column layout continues
\begin{minipage}[t]{0.30\textwidth}

% Languages & Packages
\cvsection{Languages \& Packages}
\textbf{Python} - OpenCV, MatplotLib, PyTest, Scikit Image, Scikit Learn, Seaborn, Tensorflow.\\[4pt]
\skillbar{Python}{5}

\vspace{4pt}
\textbf{Web Tech} - HTML, CSS, JavaScript (AJAX), Django\\[2pt]
\skillbar{HTML}{3}

\vspace{4pt}
\textbf{Data} - SQL, data pipeline development\\[2pt]
\skillbar{CSS}{3}

\vspace{4pt}
\textbf{Other} - MatLab, Markdown, LaTeX\\[2pt]
\skillbar{JavaScript}{4}

\vspace{12pt}

% Key Software
\cvsection{Key Software}
\textbf{Git} - Authored over 1.2k commits, 106 PR's and 72 issues in 2025. Setup automated CI/CD, test and publishing, GitHub actions.

\vspace{6pt}
\textbf{Tensorflow / DVC} - Created multiple reproducible ML training pipelines with data version control to compare parameter / data / architecture changes.

\vspace{6pt}
\textbf{High Performance Computing} - Unix, environment creation, SLURM, and parallelisation.

\vspace{6pt}
\textbf{Docker} - Making docker files and running containers for containerisation to reduce machine-to-machine variability.

\vspace{6pt}
\textbf{Testing and Documentation} - PyTest, Sphinx, environment creation, SLURM, and parallelisation.

\vspace{12pt}

% Achievements
\cvsection{Achievements}
\begin{cvlist}
    \item Invited speaker at a BIRS DNA topology conference in Canada on my novel image analysis pipeline.
    \item Awarded the 2023 Sheffield FAIR software development award.
    \item Achieved the best client satisfaction score of my analytics team at IBM.
\end{cvlist}

\end{minipage}
\hfill
\begin{tikzpicture}[baseline=(current bounding box.north)]
    \draw[lightgray, line width=0.5pt] (0,0) -- (0,-27cm);
\end{tikzpicture}
\hfill
\begin{minipage}[t]{0.62\textwidth}

% Published Packages
\cvsection{Published Packages}
\begin{cvlist}
    \item \textbf{TopoStats} - Atomic force microscopy image analysis software to quantify and characterise topographs of nanoscale biomolecules. Has $9,000+$ downloads from international research groups. Winner of the 2023 Sheffield FAIR software award.
    \item \textbf{AFMReader} - General file loader for many atomic force microscopy filetypes to extract data and metadata into Python. Has $17,000+$ downloads demonstrating scalable software adoption.
    \item \textbf{Napari-AFMReader} - A widget for the interactive BioImage viewer software "Napari" to help integrate atomic force microscopy images into the bioimage analysis community.
\end{cvlist}

\vspace{12pt}

% Relevant Work Experience
\cvsection{Relevant Work Experience}

\cvsubsection{IBM | 2018 - 2019 | Cognos Analytics Technical Support Analyst}
\begin{cvlist}
    \item Resolved 230+ cases spanning general questions, error messages, defects, and load balancing issues.
    \item Managed 10-20 concurrent clients/cases, prioritising system critical and old cases.
    \item Developed workarounds tailored to client requirements and software / hardware limitations.
    \item Configured minimal test case environments on Unix and Microsoft operating systems.
    \item Root cause analysis through the investigation of log files.
    \item Built Cognos Analytics reports using SQL queries.
    \item Authored 36 technical documents and 11 corrections.
    \item Scored the team highest net promoter score of 79 reflecting client satisfaction.
\end{cvlist}

\vspace{12pt}

% Relevant Interests and Commitments
\cvsection{Personal Projects}
\begin{cvlist}
    \item \textbf{Machine Learning}. Developed machine learning segmentation and style transfer models alongside other image processing libraries to design personalised cards and wedding invitations. Currently building into an interactive Django website.
    \item \textbf{Completed courses}. Using medical imaging deep-learning models, and contributing and developing microscopy deep-learning resources, experience with HPC resources from introductory skills courses.
    \item \textbf{Teaching \& knowledge transfer}. Beginners deep learning workshop in collaboration with the Research Software Engineering team pitched at academics to use in their own research.
    \item \textbf{Web Development (Django)} - Developed a dynamic website portfolio of interactive Python projects using the Django and AJAX frameworks on a Raspberry Pi Server.
\end{cvlist}

\end{minipage}

\end{document}