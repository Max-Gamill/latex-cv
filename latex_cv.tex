\documentclass[a4paper,11pt]{article}

% Packages
\usepackage[utf8]{inputenc}
\usepackage[T1]{fontenc}
\usepackage{geometry}
\usepackage{xcolor}
\usepackage{fontawesome5}
\usepackage{tikz}
\usepackage{graphicx}
\usepackage{enumitem}
\usepackage{tabularx}
\usepackage{array}
\usepackage{multicol}
\usepackage{textcomp}
\usepackage{url}
\usepackage{hyperref}

% Page setup
\geometry{margin=0.5in}
\pagestyle{empty}
\setlength{\parindent}{0pt}
\setlength{\parskip}{6pt}

% Colours
\definecolor{primarycolour}{RGB}{220, 57, 53}%red{36,123,160}cerulean
\definecolor{darkgray}{RGB}{80, 80, 80}
\definecolor{lightgray}{RGB}{150, 150, 150}
\definecolor{backgroundgray}{RGB}{245, 245, 245}

% Custom commands for skill bars
\newcommand{\skillbar}[2]{%
    \begin{tikzpicture}[baseline]
        \fill[lightgray] (0,0) rectangle (5,0.15);
        \fill[black] (0,0) rectangle (#2*5/5,0.15);
    \end{tikzpicture}%
}

% Hyperref
\hypersetup{colorlinks=true, urlcolor=blue} 

% Section formatting
\newcommand{\cvsection}[1]{%
    \vspace{6pt}
    {\color{primarycolour}\large\bfseries #1}
    \vspace{4pt}
    \hrule
    \vspace{4pt}
}

\newcommand{\cvsubsection}[1]{%
    \vspace{4pt}
    {\color{black}\bfseries #1}
    \vspace{2pt}
}

% Custom list environment
\newenvironment{cvlist}{%
    \begin{itemize}[leftmargin=12pt,itemsep=0pt,parsep=0pt,topsep=0pt]
}{%
    \end{itemize}
}

\begin{document}

% Header with photo and contact info
\begin{minipage}[c]{0.65\textwidth}
    {\Huge\bfseries Max Gamill, Ph.D.}\\[8pt]
    % Contact circles
    {\color{primarycolour}\Huge\textbullet\textbullet\textbullet}
\end{minipage}
%\hfill
\begin{minipage}[c]{0.32\textwidth}
    \raggedleft
    % Contact
    \cvsection{Contact}
    +44 7841 907404 \hspace{6pt}\faPhone \\ [2pt]
    maxgamill@live.com \hspace{6pt}\faEnvelope \\ [2pt]
    \href{https://www.linkedin.com/in/max-gamill/}{Max Gamill} \hspace{6pt}\faLinkedin \hspace{12pt}
    \href{https://www.linkedin.com/in/max-gamill/}{Max-Gamill} \hspace{6pt}\faGithub
\end{minipage}

\vspace{0pt}


% PAGE 1
\begin{minipage}[t]{0.32\textwidth}

    % Achievements
    \cvsection{Key Achievements}
    \begin{cvlist}
        \item Awarded the 2023 Sheffield FAIR software development award.
        \item Invited speaker at the BIRS DNA topology conference in Canada for my novel image analysis pipeline.
        \item First author Nature Communications journal paper.
        \item Achieved the highest client satisfaction score of 79\% in my analytics team at IBM.
    \end{cvlist}

    % Key Skills
    \cvsection{Key Skills}
    \cvsubsection{Production Software Development}

    | FAIR Software Development |
    | End-to-end Python Pipelines |
    | Milestone Planning | Version Control | Documentation (Sphinx, GitHub Pages) | Containerisation (Docker) | Deployment |

    \cvsubsection{Machine \& Deep Learning}

    | Scikit-Image | Scikit-Learn |
    
    | Clustering (kNN, DBSCAN, GMM) | TensorFlow | Data Version Control |
    Albumentations | 
    
    | Segmentation Models (U-Net) | 
    
    | Object Detection Models (YOLOv3, Mask-RCNN) |
    
    | Generative Models (CVAE) |

    \cvsubsection{CI/CD \& DevOps}

    | Git Collaboration | GitHub 
    
    Actions | Test Automation | Tagged Releases | PyTest | Linting |

    | Hosting Town Halls |
    
    | Leading Multi-disciplinary Teams | User Training |

    \cvsubsection{High Performance Computing}

    | Unix | Environment Management | SLURM and PBS Schedulers | MPI and Multiprocessing Parallelisation | HPC User Support |

\end{minipage}
\hfill
\begin{tikzpicture}[baseline=(current bounding box.north)]
    \draw[lightgray, line width=0.5pt] (0,0) -- (0,-24cm);
\end{tikzpicture}
\hfill
\begin{minipage}[t]{0.62\textwidth}

    % Profile
    \cvsection{Profile}
    Software developer with 5 years experience collaboratively building, testing, and enhancing research software from computer vision pipelines to Django web apps. These software packages total over $37,000$ downloads and won the 2023 Sheffield FAIR software award. Motivated to drive research further by a proven track record of resolving HPC issues and developing FAIR research software.

    % Professional Experience
    \cvsection{Professional Experience}

    \cvsubsection{\textbf{Imperial College London \hfill | 2025 - Present |}}

    \textit{HPC and RSE Experience Programme}
    \begin{cvlist}
        \item Created a Cookiecutter UV template helping initialise new projects with documentation and CI/CD system tests. 
        \item Built interactive Django projects supporting environmental and energy research data, and team resource management.
        \item Resolved bash scripting, resource availability, software and parallel processing issues on HPC systems.
        %\item Intel Project
    \end{cvlist}

    \cvsubsection{\textbf{University of Sheffield \hfill | 2021 - 2025 |}}

    \textit{Postgraduate Researcher}
    \begin{cvlist}
        \item Collaboratively developed classical and machine learning pipelines for image analysis software with $37,000+$ downloads; TopoStats, AFMReader, and Napari-AFMReader.
        \item Trained and evaluated k-means, DBSCAN, and GMM models to cluster similar DNA shapes within the latent space of a loss-function modified CVAE. The GMM achieved 60\% accuracy in a non-discrete classification task.
        \item Trained U-Net models for segmentation improvements of touching objects, reducing the error of area statistics by $\sim$30\%.
        \item Validated YOLOv3 and Mask R-CNN models to classify biomolecular structures. Identified a dataset imbalance (70\% in class 0 of 8), addressed by developing synthetic data for transfer learning.
        \item Improved pipeline governance by liaising with Microscopy companies and integrating proprietary file formats, removing bias.
        \item Guided external stakeholders to contribute and maintain software, helping with lifecycle management, and organised town halls between developers and users to align milestones to user needs.
        \item Disseminated knowledge via seminars, posters, and software workshops at international conferences (CBIAS, BIRS, I2K).
        \item First author of a Nature Communications paper - a 14.7 impact-factor journal.
    \end{cvlist}

    \cvsubsection{\textbf{IBM \hfill | 2018 - 2019 |}}

    \textit{Cognos Analytics Technical Support Analyst}
    \begin{cvlist}
        \item Resolved 230+ cases spanning general questions, errors, defects, workarounds, and load balancing issues.
        \item Authored 36 technical documents and 11 corrections.
        \item Managed 10-20 concurrent cases, prioritising system critical cases.
        \item Configured minimal test case environments on Unix and Microsoft operating systems.
        \item Root cause analysis through the investigation of log files.
        \item Scored the team highest client satisfaction score of 79.
    \end{cvlist}

\end{minipage}

\newpage

% PAGE 2 - Two-column layout continues
\begin{minipage}[t]{0.32\textwidth}

    % Languages & Packages
    \cvsection{Languages \& Packages}
    \textbf{Python}\\
    |  Pandas | Numpy | PyTest | Black | PyLint | Seaborn | Matplotlib  | \\[4pt]
    \skillbar{Python}{5}

    \vspace{4pt}
    \textbf{Python - ML \& DL}\\
    | Scikit-Image | Scikit-Learn | 
    
    | TensorFlow | Albumentations  | \\[2pt]
    \skillbar{MLDL}{4}

    \vspace{4pt}
    \textbf{Web Development}\\
    | Django | HTML | JavaScript | 
    
    | CSS  | \\[2pt]
    \skillbar{CSS}{3}

    \vspace{4pt}
    \textbf{Documentation}\\
    | Markdown | GitHub Pages |
    
    | LaTeX | \\[2pt]
    \skillbar{Other}{4}

    % Key Software
    \cvsection{Key Software}
    \cvsubsection{Git}

    Authored over 1.2k commits, 106 PR's and 72 issues in 2025. Setup automated CI/CD, test and publishing, GitHub actions.

    \cvsubsection{TensorFlow / DVC}

    Created multiple reproducible ML training pipelines with data version control to compare parameter / data / architecture changes.

    \cvsubsection{High Performance Computing}

    Unix, environment creation, SLURM, and parallelisation.

    \cvsubsection{Docker}

    Making docker files and running containers to reduce machine-to-machine variability.

    \cvsubsection{Testing and Documentation}

    PyTest (unit, integration, system testing), Sphinx, GitHub pages.

\end{minipage}
\hfill
\begin{tikzpicture}[baseline=(current bounding box.north)]
    \draw[lightgray, line width=0.5pt] (0,0) -- (0,-21cm);
\end{tikzpicture}
\hfill
\begin{minipage}[t]{0.62\textwidth}

    % Published Packages
    \cvsection{Published Packages}
    \begin{cvlist}
        \item \textbf{TopoStats} - Atomic force microscopy image analysis software to quantify and characterise topographs of nanoscale biomolecules. Has $12,000+$ downloads from international research groups. Winner of the 2023 Sheffield FAIR software award.
        \item \textbf{AFMReader} - General file loader for many atomic force microscopy file types to extract data and metadata into Python. Has $25,000+$ downloads demonstrating scalable software adoption.
        \item \textbf{Napari-AFMReader} - A widget for the interactive BioImage viewer software "Napari" to help integrate atomic force microscopy images into the bioimage analysis community. Has $450+$ downloads for recent software and is the backbone of the Napari-TopoStats software (in publication).
    \end{cvlist}

    % Education
    \cvsection{Education}

    \cvsubsection{University of Sheffield \hfill | 2021 - 2025 |}

    \textit{Doctorate of Philosophy in Computational Biophysics}

    \cvsubsection{University of Leeds \hfill | 2016 - 2021 |}

    \textit{Master of Physics | First Class}
    \begin{cvlist}
        \item Trained and evaluated a 3D-point predicting DL model was $\sim$30\% more robust to $4\times$ object density and noise vs a mathematical model. This better mapped receptor density across a SMLM killer T-Cell image.
    \end{cvlist}

    % Relevant Interests and Commitments
    \cvsection{Personal Projects \& Commitments}
    \begin{cvlist}
        \item \textbf{Machine Learning}. Developed deep learning segmentation and style transfer models alongside image processing scripts to design personalised cards. Currently building into an interactive Django website. Using medical imaging BioImage Zoo and DL4Mic deep learning models during workshops. Hosting my own LLM via LM Studio.
        \item \textbf{Web Development (Django)}. Developed a dynamic website portfolio of interactive Python projects using the Django and AJAX frameworks on a Raspberry Pi Server.
        \item \textbf{Completed Courses}. Adding New Knowledge to LLMs (hosted by NVIDIA). Accelerating model training using HPC resources after attending introductory HPC skills courses. Attended multiple Git and GitKracken courses. Contributing and developing microscopy deep learning resources at I2K and CBIAS workshops.
        \item \textbf{Teaching \& Knowledge Transfer}. Taught academics a beginners deep learning workshop in collaboration with the Research Software Engineering team. Developed best use HPC and TopoStats software guides as living documents.
    \end{cvlist}

\end{minipage}

% Referees
\cvsection{Referees}

\begin{minipage}[t]{0.45\textwidth}

    \textbf{Dr Adrian D'Alesandro}\\
    Research Software Engineer\\
    Imperial College London\\
    \faEnvelope\hspace{6pt} a.dalessandro@imperial.ac.uk

    \vspace{8pt}

    \textbf{Prof. Alice Pyne}\\
    Professor of Biophysics\\
    University of Sheffield\\
    \faEnvelope\hspace{6pt} a.l.pyne@sheffield.ac.uk

\end{minipage}
\hfill
\begin{minipage}[t]{0.45\textwidth}

    \textbf{Mr. Neil Shephard}\\
    Research Software Engineer\\
    University of Sheffield\\
    \faEnvelope\hspace{6pt} n.shephard@sheffield.ac.uk

    \vspace{8pt}

    \textbf{Dr. Laura Wiggins}\\
    Postdoctoral Researcher\\
    University of Sheffield\\
    \faEnvelope\hspace{6pt} l.wiggins@sheffield.ac.uk

\end{minipage}

\end{document}